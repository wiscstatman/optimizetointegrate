\documentclass[11pt]{article}%
\usepackage{amsfonts}
\usepackage{amssymb}
\usepackage{graphicx}
\usepackage{setspace}
\usepackage[square]{natbib}
\usepackage{amsmath}%
\usepackage{amsthm}%
\setcounter{MaxMatrixCols}{30}
\usepackage{palatino}
\usepackage{paralist}
\usepackage[top=8pc,bottom=8pc,left=8pc,right=8pc]{geometry}

%TCIDATA{OutputFilter=latex2.dll}
%TCIDATA{Version=4.10.0.2363}
%TCIDATA{CSTFile=LaTeX article.cst}
%TCIDATA{Created=Fri Dec 03 14:08:36 1999}
%TCIDATA{LastRevised=Tuesday, April 25, 2006 11:49:43}
%TCIDATA{<META NAME="GraphicsSave" CONTENT="32">}
%TCIDATA{<META NAME="DocumentShell" CONTENT="Journal Articles\Standard LaTeX Article">}
\newtheorem{theorem}{Theorem}
\newtheorem{fact}{Fact}
\newtheorem{example}{Example}
\newtheorem{acknowledgement}[theorem]{Acknowledgement}
\newtheorem{algorithm}[theorem]{Algorithm}
\newtheorem{axiom}[theorem]{Axiom}
\newtheorem{case}[theorem]{Case}
\newtheorem{claim}[theorem]{Claim}
\newtheorem{conclusion}[theorem]{Conclusion}
\newtheorem{condition}[theorem]{Condition}
\newtheorem{conjecture}[theorem]{Conjecture}
\newtheorem{corollary}[theorem]{Corollary}
\newtheorem{criterion}[theorem]{Criterion}
\newtheorem{definition}[theorem]{Definition}

\newtheorem{exercise}[theorem]{Exercise}
\newtheorem{lemma}[theorem]{Lemma}
\newtheorem{notation}[theorem]{Notation}
\newtheorem{problem}[theorem]{Problem}
\newtheorem{proposition}[theorem]{Proposition}
\newtheorem{remark}[theorem]{Remark}
\newtheorem{solution}[theorem]{Solution}
\newtheorem{summary}[theorem]{Summary}


\newcommand{\what}[1][j]{\hat\omega_{#1}^{-1(g)}}
\newcommand{\bX}{\bf{X}}
\newcommand{\Oi}{\Omega^{(g)}}
\newcommand{\bone}{\bf{1}}
\def\E{\qopname\relax o{E}}
\newcommand{\diag}{diag}
\newcommand{\PG}{ \mathcal{PG}}
\newcommand{\Ga}{ \mathcal{G} }
\newcommand{\N}{ \mathcal{N} }
\newcommand{\DE}{ \mathcal{DE} }
\newcommand{\Ex}{ \mathcal{E} }
\newcommand{\prox}{ \mathop{\mathrm{prox}} }
\newcommand{\R}{\mathcal{R}}
\newcommand{\Rbar}{ \overline{\R} }
\newcommand{\sgn}{\mathop{\mathrm{sgn}}}
\newcommand{\dd}{\mathrm{d}}
\newcommand{\enorm}[1]{\Vert #1 \Vert_2}

\newcommand{\vnorm}[1]{\left|\left|#1\right|\right|}


\newcommand{\bx}{{\bf x}}
\newcommand{\by}{{\bf y}}
\newcommand{\bs}{{\bf s}}
\newcommand{\bm}{{\bf m}}
\newcommand{\bu}{{\bf u}}
\newcommand{\bv}{{\bf v}}



\begin{document}

\section{Posterior Means and Modes}

Can we find $ \phi$, such that the Bayesian posterior estimator $ \hat{\theta} = E( \theta | y) $ with a pseudo-prior $p(\theta) $ is the same
as a solution to a $\phi$-regularised problem, namely
$$
 \hat{\theta} = E( \theta | y) = {\rm arg min}_\theta \left \{ \vnorm{y - \theta }^2 + \phi( \theta ) \right \}
$$
The rhs is known as the proximal operator of $ \phi$.

A useful result for a location problem is Tweedie's-Masreliez's formula. Efron (2011) discusses Tweedie's formula for the posterior mean in a normal location problem.
$$
\hat{\theta}(y) = E( \theta | y ) = y + \frac{d}{dy} \log m(y)
$$
where $ m(y) = \int \phi( y - \theta ) p( \theta ) d \theta $ and $ p(\theta ) \propto \exp \left ( - \phi( \theta ) \right ) $.

Strawderman et al (2015), Polson and Scott (2015) and Gribonval (2011) all discuss writing posterior means as modes.
Penalties written as envelopes and the implying mixing distribution found.
Gribonval (2011) addressed the following problem. Given any $z$, find $ y $, such that $ E(\theta | y ) = z $. 

Hence, if we want to match a mode with a mean, we can ``solve'' for the penalty, $\phi(\theta) $, given a marginal distribution, $m(y)$, via the equation for the proximal operator
$$
\nabla \phi \left ( \hat{\theta} (y) \right ) =  \frac{d}{dy} \log m(y) \; .
$$
If $ \phi $ is non-differentiable at a point, we replace $ \nabla $ by $ \partial $.

Given $ \hat{\theta}(y) $, we can solve for $ \phi( \theta ) $. Thus the posterior mean under a spike and slab prior is a posterior mode under the induced penalty--which is typically non-convex. 

\end{document}

